%% Generated by Sphinx.
\def\sphinxdocclass{report}
\documentclass[letterpaper,10pt,english]{sphinxmanual}
\ifdefined\pdfpxdimen
   \let\sphinxpxdimen\pdfpxdimen\else\newdimen\sphinxpxdimen
\fi \sphinxpxdimen=.75bp\relax
%% turn off hyperref patch of \index as sphinx.xdy xindy module takes care of
%% suitable \hyperpage mark-up, working around hyperref-xindy incompatibility
\PassOptionsToPackage{hyperindex=false}{hyperref}

\PassOptionsToPackage{warn}{textcomp}

\catcode`^^^^00a0\active\protected\def^^^^00a0{\leavevmode\nobreak\ }
\usepackage{cmap}
\usepackage{fontspec}
\defaultfontfeatures[\rmfamily,\sffamily,\ttfamily]{}
\usepackage{amsmath,amssymb,amstext}
\usepackage{polyglossia}
\setmainlanguage{english}






\usepackage[Bjornstrup]{fncychap}
\usepackage{sphinx}

\fvset{fontsize=\small}
\usepackage{geometry}


% Include hyperref last.
\usepackage{hyperref}
% Fix anchor placement for figures with captions.
\usepackage{hypcap}% it must be loaded after hyperref.
% Set up styles of URL: it should be placed after hyperref.
\urlstyle{same}


\usepackage{sphinxmessages}



\usepackage[titles]{tocloft}
\cftsetpnumwidth {1.25cm}\cftsetrmarg{1.5cm}
\setlength{\cftchapnumwidth}{0.75cm}
\setlength{\cftsecindent}{\cftchapnumwidth}
\setlength{\cftsecnumwidth}{1.25cm}


\title{SingleCellData}
\date{Jun 27, 2020}
\release{1.0.0}
\author{Edwin Vans}
\newcommand{\sphinxlogo}{\vbox{}}
\renewcommand{\releasename}{Release}
\makeindex
\begin{document}

\pagestyle{empty}
\sphinxmaketitle
\pagestyle{plain}
\sphinxtableofcontents
\pagestyle{normal}
\phantomsection\label{\detokenize{index::doc}}



\chapter{SingleCell Class}
\label{\detokenize{index:module-singlecelldata}}\label{\detokenize{index:singlecell-class}}\index{module@\spxentry{module}!singlecelldata@\spxentry{singlecelldata}}\index{singlecelldata@\spxentry{singlecelldata}!module@\spxentry{module}}

\section{SingleCellData}
\label{\detokenize{index:singlecelldata}}
A Python package that contains a class for managing single\sphinxhyphen{}cell RNA\sphinxhyphen{}seq datasets.
See \sphinxurl{https://edwinv87.github.io/singlecelldata/} for more information and documentation.
\index{SingleCell (class in singlecelldata)@\spxentry{SingleCell}\spxextra{class in singlecelldata}}

\begin{fulllineitems}
\phantomsection\label{\detokenize{index:singlecelldata.SingleCell}}\pysiglinewithargsret{\sphinxbfcode{\sphinxupquote{class }}\sphinxcode{\sphinxupquote{singlecelldata.}}\sphinxbfcode{\sphinxupquote{SingleCell}}}{\emph{\DUrole{n}{dataset}}, \emph{\DUrole{n}{data}}, \emph{\DUrole{n}{celldata}\DUrole{o}{=}\DUrole{default_value}{None}}, \emph{\DUrole{n}{genedata}\DUrole{o}{=}\DUrole{default_value}{None}}}{}
A python class for managing single\sphinxhyphen{}cell RNA\sphinxhyphen{}seq datasets.
\index{dataset (singlecelldata.SingleCell attribute)@\spxentry{dataset}\spxextra{singlecelldata.SingleCell attribute}}

\begin{fulllineitems}
\phantomsection\label{\detokenize{index:singlecelldata.SingleCell.dataset}}\pysigline{\sphinxbfcode{\sphinxupquote{dataset}}}
A string for the name of the dataset.
\begin{quote}\begin{description}
\item[{Type}] \leavevmode
str

\end{description}\end{quote}

\end{fulllineitems}

\index{data (singlecelldata.SingleCell attribute)@\spxentry{data}\spxextra{singlecelldata.SingleCell attribute}}

\begin{fulllineitems}
\phantomsection\label{\detokenize{index:singlecelldata.SingleCell.data}}\pysigline{\sphinxbfcode{\sphinxupquote{data}}}
The main dataframe or assay for storing the gene expression counts. The shape of this
dataframe is (d x n) where d is the number of genes (features) and n is the
number of cells (samples).
\begin{quote}\begin{description}
\item[{Type}] \leavevmode
Pandas Dataframe

\end{description}\end{quote}

\end{fulllineitems}

\index{celldata (singlecelldata.SingleCell attribute)@\spxentry{celldata}\spxextra{singlecelldata.SingleCell attribute}}

\begin{fulllineitems}
\phantomsection\label{\detokenize{index:singlecelldata.SingleCell.celldata}}\pysigline{\sphinxbfcode{\sphinxupquote{celldata}}}
The dataframe or assay used to store more data (metadata) about cells. The shape of this dataframe
is (n x m), where m is number of columns representing different types of information
about the cells, such as cell types etc.
\begin{quote}\begin{description}
\item[{Type}] \leavevmode
Pandas Dataframe

\end{description}\end{quote}

\end{fulllineitems}

\index{genedata (singlecelldata.SingleCell attribute)@\spxentry{genedata}\spxextra{singlecelldata.SingleCell attribute}}

\begin{fulllineitems}
\phantomsection\label{\detokenize{index:singlecelldata.SingleCell.genedata}}\pysigline{\sphinxbfcode{\sphinxupquote{genedata}}}
The dataframe or assay used to store more data (metadata) about genes. The shape of this dataframe
is (d x m), where m is number of columns representing different types of information
about the genes such as gene names etc.
\begin{quote}\begin{description}
\item[{Type}] \leavevmode
Pandas Dataframe

\end{description}\end{quote}

\end{fulllineitems}

\index{dim (singlecelldata.SingleCell attribute)@\spxentry{dim}\spxextra{singlecelldata.SingleCell attribute}}

\begin{fulllineitems}
\phantomsection\label{\detokenize{index:singlecelldata.SingleCell.dim}}\pysigline{\sphinxbfcode{\sphinxupquote{dim}}}
Variable representing the dimensionality of the data assay. It is (d, n).
\begin{quote}\begin{description}
\item[{Type}] \leavevmode
tuple

\end{description}\end{quote}

\end{fulllineitems}

\index{addCellData() (singlecelldata.SingleCell method)@\spxentry{addCellData()}\spxextra{singlecelldata.SingleCell method}}

\begin{fulllineitems}
\phantomsection\label{\detokenize{index:singlecelldata.SingleCell.addCellData}}\pysiglinewithargsret{\sphinxbfcode{\sphinxupquote{addCellData}}}{\emph{\DUrole{n}{col\_data}}, \emph{\DUrole{n}{col\_name}}}{}
Adds a column in the celldata dataframe.
\begin{quote}\begin{description}
\item[{Parameters}] \leavevmode\begin{itemize}
\item {} 
\sphinxstyleliteralstrong{\sphinxupquote{col\_data}} (\sphinxstyleliteralemphasis{\sphinxupquote{List}}\sphinxstyleliteralemphasis{\sphinxupquote{ or }}\sphinxstyleliteralemphasis{\sphinxupquote{Numpy array}}) – The data to be added to the celldata dataframe. The size of the List
or Numpy array should be equal to n.

\item {} 
\sphinxstyleliteralstrong{\sphinxupquote{col\_name}} (\sphinxstyleliteralemphasis{\sphinxupquote{str}}) – The name of the data column.

\end{itemize}

\end{description}\end{quote}

\end{fulllineitems}

\index{addGeneData() (singlecelldata.SingleCell method)@\spxentry{addGeneData()}\spxextra{singlecelldata.SingleCell method}}

\begin{fulllineitems}
\phantomsection\label{\detokenize{index:singlecelldata.SingleCell.addGeneData}}\pysiglinewithargsret{\sphinxbfcode{\sphinxupquote{addGeneData}}}{\emph{\DUrole{n}{col\_data}}, \emph{\DUrole{n}{col\_name}}}{}
Adds a column in the genedata dataframe.
\begin{quote}\begin{description}
\item[{Parameters}] \leavevmode\begin{itemize}
\item {} 
\sphinxstyleliteralstrong{\sphinxupquote{col\_data}} (\sphinxstyleliteralemphasis{\sphinxupquote{List}}\sphinxstyleliteralemphasis{\sphinxupquote{ or }}\sphinxstyleliteralemphasis{\sphinxupquote{Numpy array}}) – The data to be added to the genedata dataframe. The size of the List
or Numpy array should be equal to d.

\item {} 
\sphinxstyleliteralstrong{\sphinxupquote{col\_name}} (\sphinxstyleliteralemphasis{\sphinxupquote{str}}) – The name of the data column.

\end{itemize}

\end{description}\end{quote}

\end{fulllineitems}

\index{checkCellData() (singlecelldata.SingleCell method)@\spxentry{checkCellData()}\spxextra{singlecelldata.SingleCell method}}

\begin{fulllineitems}
\phantomsection\label{\detokenize{index:singlecelldata.SingleCell.checkCellData}}\pysiglinewithargsret{\sphinxbfcode{\sphinxupquote{checkCellData}}}{\emph{\DUrole{n}{column}}}{}
Checks whether a column exists in the celldata dataframe.
\begin{quote}\begin{description}
\item[{Parameters}] \leavevmode
\sphinxstyleliteralstrong{\sphinxupquote{column}} (\sphinxstyleliteralemphasis{\sphinxupquote{str}}) – The name of the column.

\item[{Returns}] \leavevmode
True if column exists in the dataframe, False otherwise.

\item[{Return type}] \leavevmode
bool

\end{description}\end{quote}

\end{fulllineitems}

\index{checkGeneData() (singlecelldata.SingleCell method)@\spxentry{checkGeneData()}\spxextra{singlecelldata.SingleCell method}}

\begin{fulllineitems}
\phantomsection\label{\detokenize{index:singlecelldata.SingleCell.checkGeneData}}\pysiglinewithargsret{\sphinxbfcode{\sphinxupquote{checkGeneData}}}{\emph{\DUrole{n}{column}}}{}
Checks whether a column exists in the genedata dataframe.
\begin{quote}\begin{description}
\item[{Parameters}] \leavevmode
\sphinxstyleliteralstrong{\sphinxupquote{column}} (\sphinxstyleliteralemphasis{\sphinxupquote{str}}) – The name of the column.

\item[{Returns}] \leavevmode
True if column exists in the dataframe, False otherwise.

\item[{Return type}] \leavevmode
bool

\end{description}\end{quote}

\end{fulllineitems}

\index{getCellData() (singlecelldata.SingleCell method)@\spxentry{getCellData()}\spxextra{singlecelldata.SingleCell method}}

\begin{fulllineitems}
\phantomsection\label{\detokenize{index:singlecelldata.SingleCell.getCellData}}\pysiglinewithargsret{\sphinxbfcode{\sphinxupquote{getCellData}}}{\emph{\DUrole{n}{column}}}{}
Returns data stored in the celldata dataframe.
\begin{quote}\begin{description}
\item[{Parameters}] \leavevmode
\sphinxstyleliteralstrong{\sphinxupquote{column}} (\sphinxstyleliteralemphasis{\sphinxupquote{str}}) – The name of the data column.

\item[{Returns}] \leavevmode
A n\sphinxhyphen{}dimensional array containing cell data by the column name.

\item[{Return type}] \leavevmode
Numpy array

\item[{Raises}] \leavevmode
\sphinxstyleliteralstrong{\sphinxupquote{ValueError}} – If column does not exist in the celldata dataframe.

\end{description}\end{quote}

\end{fulllineitems}

\index{getCounts() (singlecelldata.SingleCell method)@\spxentry{getCounts()}\spxextra{singlecelldata.SingleCell method}}

\begin{fulllineitems}
\phantomsection\label{\detokenize{index:singlecelldata.SingleCell.getCounts}}\pysiglinewithargsret{\sphinxbfcode{\sphinxupquote{getCounts}}}{}{}
Returns a Numpy array of the counts/data in the data dataframe. This method is called from the
class instance and requires no input arguments.
\begin{quote}\begin{description}
\item[{Returns}] \leavevmode
A (d x n) array of gene expression counts/data.

\item[{Return type}] \leavevmode
Numpy array

\end{description}\end{quote}

\end{fulllineitems}

\index{getDistinctCellTypes() (singlecelldata.SingleCell method)@\spxentry{getDistinctCellTypes()}\spxextra{singlecelldata.SingleCell method}}

\begin{fulllineitems}
\phantomsection\label{\detokenize{index:singlecelldata.SingleCell.getDistinctCellTypes}}\pysiglinewithargsret{\sphinxbfcode{\sphinxupquote{getDistinctCellTypes}}}{\emph{\DUrole{n}{column}}}{}
Returns the unique cell type information stored in the celldata dataframe.
\begin{quote}\begin{description}
\item[{Parameters}] \leavevmode
\sphinxstyleliteralstrong{\sphinxupquote{column}} (\sphinxstyleliteralemphasis{\sphinxupquote{str}}) – This parameter is the column name of the cell labels in the celldata assay.

\item[{Returns}] \leavevmode
Containing unique values in the celldata dataframe under the column passed into
this function.

\item[{Return type}] \leavevmode
Numpy array

\end{description}\end{quote}

\end{fulllineitems}

\index{getGeneData() (singlecelldata.SingleCell method)@\spxentry{getGeneData()}\spxextra{singlecelldata.SingleCell method}}

\begin{fulllineitems}
\phantomsection\label{\detokenize{index:singlecelldata.SingleCell.getGeneData}}\pysiglinewithargsret{\sphinxbfcode{\sphinxupquote{getGeneData}}}{\emph{\DUrole{n}{column}}}{}
Returns data stored in the genedata dataframe.
\begin{quote}\begin{description}
\item[{Parameters}] \leavevmode
\sphinxstyleliteralstrong{\sphinxupquote{column}} (\sphinxstyleliteralemphasis{\sphinxupquote{str}}) – The name of the data column.

\item[{Returns}] \leavevmode
A d\sphinxhyphen{}dimensional array containing gene data by the column name.

\item[{Return type}] \leavevmode
Numpy array

\item[{Raises}] \leavevmode
\sphinxstyleliteralstrong{\sphinxupquote{ValueError}} – If column does not exist in the genedata dataframe.

\end{description}\end{quote}

\end{fulllineitems}

\index{getNumericCellLabels() (singlecelldata.SingleCell method)@\spxentry{getNumericCellLabels()}\spxextra{singlecelldata.SingleCell method}}

\begin{fulllineitems}
\phantomsection\label{\detokenize{index:singlecelldata.SingleCell.getNumericCellLabels}}\pysiglinewithargsret{\sphinxbfcode{\sphinxupquote{getNumericCellLabels}}}{\emph{\DUrole{n}{column}}}{}
Returns the numeric (int) cell labels from the celldata assay which contains the string or int cell labels.
This method is useful when computing Rand Index or Adjusted Rand Index after clustering.
\begin{quote}\begin{description}
\item[{Parameters}] \leavevmode
\sphinxstyleliteralstrong{\sphinxupquote{column}} (\sphinxstyleliteralemphasis{\sphinxupquote{str}}) – This parameter is the column name of the string or int cell labels in the celldata assay.

\item[{Returns}] \leavevmode
Array containing the integer representation of data in the celldata dataframe under the
column passed into this function.

\item[{Return type}] \leavevmode
Numpy array (int)

\end{description}\end{quote}

\end{fulllineitems}

\index{isSpike() (singlecelldata.SingleCell method)@\spxentry{isSpike()}\spxextra{singlecelldata.SingleCell method}}

\begin{fulllineitems}
\phantomsection\label{\detokenize{index:singlecelldata.SingleCell.isSpike}}\pysiglinewithargsret{\sphinxbfcode{\sphinxupquote{isSpike}}}{\emph{\DUrole{n}{spike\_type}}, \emph{\DUrole{n}{gene\_names\_column}}}{}
Prints a message if spike\sphinxhyphen{}ins are detected in the dataset. Creates
a filter to remove spike\sphinxhyphen{}ins from the dataset when counts/data is
returned using getCounts() method.
\begin{quote}\begin{description}
\item[{Parameters}] \leavevmode
\sphinxstyleliteralstrong{\sphinxupquote{spike\_type}} (\sphinxstyleliteralemphasis{\sphinxupquote{str}}) – A string representing the type of spike\sphinxhyphen{}in.

\end{description}\end{quote}

\end{fulllineitems}

\index{print() (singlecelldata.SingleCell method)@\spxentry{print()}\spxextra{singlecelldata.SingleCell method}}

\begin{fulllineitems}
\phantomsection\label{\detokenize{index:singlecelldata.SingleCell.print}}\pysiglinewithargsret{\sphinxbfcode{\sphinxupquote{print}}}{}{}
Prints a summary of the single\sphinxhyphen{}cell dataset.

\end{fulllineitems}

\index{removeCellData() (singlecelldata.SingleCell method)@\spxentry{removeCellData()}\spxextra{singlecelldata.SingleCell method}}

\begin{fulllineitems}
\phantomsection\label{\detokenize{index:singlecelldata.SingleCell.removeCellData}}\pysiglinewithargsret{\sphinxbfcode{\sphinxupquote{removeCellData}}}{\emph{\DUrole{n}{column}}}{}
Removes a column from the celldata dataframe. First checks
whether the column exists in the celldata dataframe.
\begin{quote}\begin{description}
\item[{Parameters}] \leavevmode
\sphinxstyleliteralstrong{\sphinxupquote{column}} (\sphinxstyleliteralemphasis{\sphinxupquote{str}}) – The name of the data column.

\end{description}\end{quote}

\end{fulllineitems}

\index{removeGeneData() (singlecelldata.SingleCell method)@\spxentry{removeGeneData()}\spxextra{singlecelldata.SingleCell method}}

\begin{fulllineitems}
\phantomsection\label{\detokenize{index:singlecelldata.SingleCell.removeGeneData}}\pysiglinewithargsret{\sphinxbfcode{\sphinxupquote{removeGeneData}}}{\emph{\DUrole{n}{column}}}{}
Removes a column from the genedata dataframe. First checks
whether the column exists in the genedata dataframe.
\begin{quote}\begin{description}
\item[{Parameters}] \leavevmode
\sphinxstyleliteralstrong{\sphinxupquote{column}} (\sphinxstyleliteralemphasis{\sphinxupquote{str}}) – The name of the data column.

\end{description}\end{quote}

\end{fulllineitems}

\index{setCounts() (singlecelldata.SingleCell method)@\spxentry{setCounts()}\spxextra{singlecelldata.SingleCell method}}

\begin{fulllineitems}
\phantomsection\label{\detokenize{index:singlecelldata.SingleCell.setCounts}}\pysiglinewithargsret{\sphinxbfcode{\sphinxupquote{setCounts}}}{\emph{\DUrole{n}{new\_counts}}}{}
Sets the new counts values in the data dataframe.
\begin{quote}\begin{description}
\item[{Parameters}] \leavevmode
\sphinxstyleliteralstrong{\sphinxupquote{new\_counts}} (\sphinxstyleliteralemphasis{\sphinxupquote{Numpy array}}) – A numpy array with the shape = dim, representing new count values. The data dataframe will
be updated with the new count values.

\end{description}\end{quote}

\end{fulllineitems}


\end{fulllineitems}



\renewcommand{\indexname}{Python Module Index}
\begin{sphinxtheindex}
\let\bigletter\sphinxstyleindexlettergroup
\bigletter{s}
\item\relax\sphinxstyleindexentry{singlecelldata}\sphinxstyleindexpageref{index:\detokenize{module-singlecelldata}}
\end{sphinxtheindex}

\renewcommand{\indexname}{Index}
\footnotesize\raggedright\printindex
\end{document}